\documentclass[10pt,a4paper]{article}
\usepackage[utf8]{inputenc}
\usepackage[italian]{babel}
\usepackage{amsmath}
\usepackage{xcolor}
\usepackage{siunitx}
\usepackage{circuitikz}
\usepackage{amsfonts}
\usepackage{amssymb}
\usepackage{graphicx}
\usepackage[left=2cm,right=2cm,top=2cm,bottom=2cm]{geometry}
\newcommand{\rem}[1]{[\emph{#1}]}
\newcommand{\exn}{\phantom{xxx}}
\renewcommand{\thesubsection}{\thesection.\alph{subsection}}  %% use 1.a numbering

\author{Gruppo 1G.BN \\ Massimo Bilancioni, Alessandro Foligno, Giuseppe Zanichelli }
\title{Es06B:Usi non lineari dell’ OpAmp }
\begin{document}
	\date{8 novembre 2018}
	\maketitle
	
	
	\section*{Scopo dell' esperienza}

	Scopo dell'esperienza è l'analisi di circuiti che fanno usi non lineari dell'Amplificatore Operazionale




\section{Ampificatori di Carica}
	\subsection{Montaggio circuito}
		per avere la soglia a circa 200 \si{\milli \volt} si imposta il potenziometro a circa $(0.5+0.2/15)=0.51$ volte il suo valore massimo.
	\subsection{Analisi dei segnali}
		Studiando il circuito, ci si aspetta, in $V_{disc}$ un'onda quadra di duty cycle variabile e ampiezza data dalle tensioni di alimentazioni dell'op-amp, dove il segnale alto indica quando il segnale in $V_{sh}$ è sopra la soglia.
		Per quanto riguarda la prima parte del circuito, anzitutto osserviamo che, ad ogni cambio di polarità dell'onda quadra, viene iniettata nel circuito una quantità di carica pari a $Q=V_{pp} C_T$. Questo "impulso" di corrente deve finire sul condensatore $C_F$ dato che sulla resistenza non c'è abbastanza ddP per farlo passare
\section{ MULTIVIBRATORI }






\end{document}          
