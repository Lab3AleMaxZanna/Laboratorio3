\documentclass[10pt,a4paper]{article}
\usepackage[utf8]{inputenc}
\usepackage[italian]{babel}
\usepackage{amsmath}
\usepackage{xcolor}
\usepackage{siunitx}
\usepackage{amsfonts}
\usepackage{amssymb}
\usepackage{graphicx}
\usepackage{siunitx}
\usepackage[left=2cm,right=2cm,top=2cm,bottom=2cm]{geometry}
\newcommand{\rem}[1]{[\emph{#1}]}
\newcommand{\exn}{\phantom{xxx}}
\renewcommand{\thesubsection}{\thesection.\alph{subsection}}  %% use 1.a numbering

\author{Gruppo 1G.BN \\ Massimo Bilancioni, Alessandro Foligno, Giuseppe Zanichelli }
\title{Es06B:Usi non lineari dell’ OpAmp }
\begin{document}
	\date{8 novembre 2018}
	\maketitle
	
	
	\section*{Scopo dell' esperienza}
	Scopo dell'esperienza è l'analisi di circuiti che fanno usi non lineari dell'Amplificatore Operazionale




\section{Ampificatori di Carica}
	\subsection{Montaggio circuito e valore delle componenti}
		Si monta il circuito sulla basetta.
		I valori delle varie componenti sono i seguenti:
		\\$C_T=0.967\pm 0.04$ \si{\nano\farad}\\$C_f=0.958\pm 0.04$ \si{\nano\farad}\\$R_2=99.9\pm 0.9 $ \si{\ohm}\\$R_1=0.392 \pm0.03   $ \si{\mega \ohm}\\
				Il potenziometro, invece, ha una resistenza massima di circa 1\si{\mega\ohm}
	\subsection{Analisi dei segnali}
		Studiando il circuito, ci si aspetta, in $V_{disc}$ un'onda quadra di duty cycle variabile e ampiezza data dalle tensioni di alimentazioni dell'op-amp, dove il segnale alto indica quando il segnale in $V_{sh}$ è sopra la soglia.
		Per quanto riguarda la prima parte del circuito, anzitutto osserviamo che, ad ogni cambio di polarità dell'onda quadra, viene iniettata nel circuito una quantità di carica pari a $Q=V_{pp} C_T$. Questo "impulso" di corrente deve tutto sul condensatore $C_F$ dato che sulla resistenza non c'è abbastanza ddP per farlo passare.
		Dopo che il condensatore riceve la carica inizia a scaricarsi. Ci si aspetta, quindi, come ampiezza massima del segnale $V_{sh}$, $V_{max}=\frac{Q}{C_F}=2 V_{in} \frac{C_T}{C_F}\approx 2 V_{in} $ dato che le capacità sono più o meno uguali. Infatti, con un'onda quadra in ingresso $V_in\approx3V$ si osserva il segnale $V_{sh}$ riportato in Figura \ref{fig:sample}, di ampiezza $\approx 6V$.
		\begin{figure}
			\centering
			\includegraphics{sample.png}
			\caption{segnale $V_{sh}$}
			\label{fig:sample}
		\end{figure}
	
\clearpage	
\section{ MULTIVIBRATORI }


\subsection{Funzionamento}
I valori misurati di $R_1$, $R_2$ e $R_3$ sono:
\[ R_1 = (9.75\pm 0.08)\si{\kilo\ohm} \qquad  R_2 = (9.98 \pm 0.08)\si{\kilo\ohm} \qquad   R_3 = ( 0.994 \pm0.001 ) \si{\kilo\ohm}\]
  Il circuito ha 2 stati possibili: condensatore carico e condensatore scarico. L'OpAmp è configrato come un invertitore e quindi nessuno dei due stati è stabile: quando il voltaggio al capo del condensatore ha raggiunto il livello dato dal partitore di tensione $R_1/R_2$ il sistema passa nell'altro stato, con un tempo caratteristico dato dallo slew rate dell'OpAmp, dell'ordine di $10 \si{\volt\per\micro\second}$. Essendo i voltaggi usati nelle decine di volt non dovrebbe dare problemi fino a frequenze dell'ordine del centinaio di kilohertz.\label{slew_rate_count}
Il periodo di oscillazione del circuito è : \[ T = 2 RC \log( 1+ 2 R_2/R_1) \approxeq 2.23 RC\]
\subsection{Montaggio circuito}
Per ottenere un periodo di $\approx 2 \si{\milli\second}$ abbiamo scelto i seguenti valori per la resistenza e la capacità:
\[ R = ( 46.0 \pm0.5 )\si{\kilo\ohm} \qquad   C = (21.3\pm0.9 )\si{\nano \farad}\]

a cui corrisponde un periodo di oscillazione teorico:
\[T_{att}= (2.18\pm 0.14 )\si{\milli \second}\]
Inoltre abbiamo aggiunto il resistore variabile in serie a R, lasciandone un pin libero. In questo modo abbiamo potuto variare R a piacimento tra $46 \si{\kilo\ohm}$ e $146 \si{\kilo\ohm}$ per apprezzare la variazione della funzione d'onda risultante.


\subsection{Osservazioni del circuito}
Abbiamo collegato le uscite della basetta a $V_{out}, V_{+}, V_{-}$. Nel seguito potete vedere il segnale in uscita (Figura \ref{fig:V_out}) e un confronto tra $V_{+}$ e $V_{-}$. Notare come il cambio di stato sia quasi istantaneo e avvenga ogni volta che essi son uguali.
 
 \begin{figure}[h]
 	\centering
 		\includegraphics[scale=0.8]{vout_punto2.png}
 		\caption{\small segnale $V_{out}$ del multivibratore astabile }
 		\label{fig:V_out}
 \end{figure}
 
 \begin{figure}[h]
 	\centering
 		\includegraphics[scale=0.8]{v+_v-.png}
 		\caption{\small i segnali $V_+$( in azzurro ) e $ V_-$ ( in arancione )  multivibratore astabile}
 		\label{fig:V+V-}
 \end{figure}
Le misure picco picco per  segnali  $V_{out}, v_{+}$ e $v_{-}$ sono :
\[ V_{out,pp}= (13.6\pm 0.2)\si{\kilo\ohm} \qquad  v_{+,pp}= (6.88 \pm 0.12)\si{\kilo\ohm} \qquad   v_{-,pp}= ( 7.04 \pm0.08 ) \si{\kilo\ohm}\]
Il periodo misurato dell'oscillazione misurato sull'oscilloscopio da come risultato :
\[ T = (2.26 \pm 0.02)\]
in accordo con il periodo teorico calcolato in precedenza.

\subsection{Motivazione degli Zener} 
La serie dei due  diodi  Zener back-to-back limita l'escursione in tensione, infatti deve essere $|V_{out}| \le V_{\gamma} +V_{z} = V_{Max}$. In questo modo il voltaggio in uscita ha un plateau a $V_{out} = \pm V_{max}$ generando l'onda quadra misurata. Dato che la resistenza dei diodi è trascurabile si rischia che l'OpAmp superi la sua potenza massima, essendo effettivamente cortocircuitato a terra. $R_3$ evita proprio questo limitando la corrente in uscita.

Dalla misura risulta \[V_{max} = ( 6.8\pm 0.1) \si \volt\]

\subsection{Periodo dell'onda quadra in uscita}
Il periodo dell'onda quadra in uscita è indipendente dal voltaggio applicato. Infatti sia nel caso in cui il dispositivo sia limitato dai due diodi Zener o in cui sia limitato dalla tensione di alimentazione dell'OpAmp il condensatore si carica con un tempo caratteristico indipendente dall'ampiezza ($1/RC$). L'OpAmp scatta quando $V_- < V_+$, ovvero:
\[V_{out} e^{\frac{t}{RC}} = (1+ 2 R_2/R_1) V_{out}\]
condizione che è indipendente dall'ampiezza di $V_{out}$, che sarà uguale a $V_{in}$ o $V_{max}$ a seconda di quale sia minore.
Una dipendenza dall'ampiezza appare solo in caso che lo slew-rate non sia trascurabile. In tal caso $V_-$ non è costante nel tempo e il tempo di carica del condensatore dipende anche da $R_{slew rate} / V_{out}$ (vedi figura \ref{fig:slew_rate}). 

\subsection{Massima frequenza}
Dal grafico sembrerebbe che non ci sia limite teorico a quanto possiamo rendere piccolo $RC$, e quindi alta la frequenza. Tuttavia due effetti entrano in gioco a limitare questa possibilità: lo slew-rate dell'OpAmp e la capacità e resistenza della basetta. Come fatto notare in \ref{slew_rate_count} lo slew rate non è apprezzabile fino a valori di $1/RC$ nelle centinaia di kilohertz. Arrivati nei megaohm il segnale in uscita non presenta più alcuna caratterisca delle onde quadre.

 \begin{figure}[h]
	\centering
	\includegraphics[scale=0.8]{freqRbassa.png}
	\caption{\small i segnali $V_+$( in azzurro ) e $V_-$ ( in arancione )  in caso di $RC$ molto basso}
	\label{fig:slew_rate}
\end{figure}

Un'altro effetto limitante la frequenza è l'impossibilità di abbassare RC a piacere, a causa della capacità della basetta. Tentando infatti di abbassare $C$ la frequenza non sale all'infinito ma ha un comportamento asintotico. Inoltre per bassi valori di $C$ il dispositivo diventa tanto sensibile all'ambiente che è possibile cambiare la frequenza in modo significativo solo passando una mano sopra di esso.

\end{document}          
