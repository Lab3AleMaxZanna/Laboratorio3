\documentclass[10pt,a4paper]{article}
\usepackage[utf8]{inputenc}
\usepackage[italian]{babel}
\usepackage{amsmath}
\usepackage{xcolor}
\usepackage{siunitx}
\usepackage{amsfonts}
\usepackage{amssymb}
\usepackage{graphicx}
\usepackage{siunitx}
\usepackage[left=2cm,right=2cm,top=2cm,bottom=2cm]{geometry}
\newcommand{\rem}[1]{[\emph{#1}]}
\newcommand{\exn}{\phantom{xxx}}
\renewcommand{\thesubsection}{\thesection.\alph{subsection}}  %% use 1.a numbering

\author{Gruppo 1G.BN \\ Massimo Bilancioni, Alessandro Foligno, Giuseppe Zanichelli }
\title{Es06B:Usi non lineari dell’ OpAmp }
\begin{document}
	\date{8 novembre 2018}
	\maketitle
	
	
	\section*{Scopo dell' esperienza}
	Scopo dell'esperienza è l'analisi di circuiti che fanno usi non lineari dell'Amplificatore Operazionale




\section{Ampificatori di Carica}
	\subsection{Montaggio circuito}
		per avere la soglia a circa 200 \si{\milli \volt} si imposta il potenziometro a circa $(0.5+0.2/15)=0.51$ volte il suo valore massimo.
	\subsection{Analisi dei segnali}
		Studiando il circuito, ci si aspetta, in $V_{disc}$ un'onda quadra di duty cycle variabile e ampiezza data dalle tensioni di alimentazioni dell'op-amp, dove il segnale alto indica quando il segnale in $V_{sh}$ è sopra la soglia.
		Per quanto riguarda la prima parte del circuito, anzitutto osserviamo che, ad ogni cambio di polarità dell'onda quadra, viene iniettata nel circuito una quantità di carica pari a $Q=V_{pp} C_T$. Questo "impulso" di corrente deve finire sul condensatore $C_F$ dato che sulla resistenza non c'è abbastanza ddP per farlo passare
\section{ MULTIVIBRATORI }


\subsection{Funzionamento}
I valori misurati di $R_1$, $R_2$ e $R_3$ sono:
\[ R_1 = (9.75\pm 0.08)\si{\kilo\ohm} \qquad  R_2 = (9.98 \pm 0.08)\si{\kilo\ohm} \qquad   R_3 = ( 0.994 \pm0.001 ) \si{\kilo\ohm}\]
  Il circuito ha 2 stati possibili: condensatore carico e condensatore scarico. L'OpAmp è configrato come un invertitore e quindi nessuno dei due stati è stabile: quando il voltaggio al capo del condensatore ha raggiunto il livello dato dal partitore di tensione $R_1/R_2$ il sistema passa nell'altro stato, con un tempo caratteristico dato dallo slew rate dell'OpAmp, dell'ordine di $10 \si{\volt\per\micro\sec}$. Essendo i voltaggi usati nelle decine di volt non dovrebbe dare problemi fino a frequenze dell'ordine del centinaio di kilohertz.
Il periodo di oscillazione del circuito è : \[ T = 2 RC \log( 1+ 2 R_2/R_1) \approxeq 2.23 RC\]
\subsection{Montaggio circuito}
Per ottenere un periodo di $\approx 2 \si{\milli\second}$ abbiamo scelto i seguenti valori per la resistenza e la capacità:
\[ R = ( 46.0 \pm0.5 )\si{\kilo\ohm} \qquad   C = (21.3\pm0.9 )\si{\nano \farad}\]

a cui corrisponde un periodo di oscillazione teorico:
\[T_{att}= (2.18\pm 0.14 )\si{\milli \second}\]
Inoltre abbiamo aggiunto il resistore variabile in serie a R, lasciandone un pin libero. In questo modo abbiamo potuto variare R a piacimento tra $46 \si{\kilo\ohm}$ e $146 \si{\kilo\ohm}$ per apprezzare la variazione della funzione d'onda risultante.



c) \rem{ foto oscilloscopio Vout v+ e v- , misurare periodo oscill,  vout = ddp clamp morse,v+ e v- insieme per vedere confronto ampiezze}
Le misure picco picco per  segnali  $V_{out}, v_{+}$ e $v_{-}$ sono :
\[ V_{out,pp}= (13.6\pm 0.2)\si{\kilo\ohm} \qquad  v_{+,pp}= (6.88 \pm 0.12)\si{\kilo\ohm} \qquad   v_{-,pp}= ( 7.04 \pm0.08 ) \si{\kilo\ohm}\]
Il periodo misurato dell'oscillazione misurato sull'oscilloscopio da come risultato :
\[ T = (2.26 \pm 0.02)\]




\begin{figure}[h]
	\begin{center}
		
			\includegraphics[scale=0.8]{v+_v-.png}
		\caption{\small i segnali $V_+$( in azzurro ) e $ V_-$ ( in arancione )  multivibratore astabile}

		\label{fig:v+v-}
	\end{center}

\end{figure}



\begin{figure}[h]
	\begin{center}
		
			\includegraphics[scale=0.8]{vout_punto2.png}
		\caption{\small segnale $V_{out}$ del multivibratore astabile }

		\label{fig:v+v-}
	\end{center}

\end{figure}


d) La serie dei due  diodi  Zener limita l'escursione in tensione, infatti deve essere $|V_{out}| \le V_{\gamma} +V_{z} = V_{Max}$.
 Se $V_{1}$ (segnale all'uscita dell'Op-Amp) fosse maggiore dii $V_{max}$ si avrebbe un cortocircuito; per evitare questa eventualità si introduce  una resistenza $R_3$ che limita la corrente.

Dalla misura risulta \[V_{max} = ( 6.8\pm 0.1) \si \volt\]






\end{document}          
