\documentclass[10pt,a4paper]{article}
\usepackage[utf8]{inputenc}
\usepackage[italian]{babel}
\usepackage{amsmath}
\usepackage{amsfonts}
\usepackage{amssymb}
\usepackage{graphicx}
\usepackage[left=2cm,right=2cm,top=2cm,bottom=2cm]{geometry}
\newcommand{\rem}[1]{[\emph{#1}]}
\newcommand{\exn}{\phantom{xxx}}
\renewcommand{\thesubsection}{\thesection.\alph{subsection}}  %% use 1.a numbering

\author{Gruppo 1x.By \\ Mario Rossi, Anna Bianchi \rem{non dimenticate i nomi}}
\title{Es05B: Circuiti lineari con Amplificatori Operazionali}
\begin{document}
	\date{23 ottembre 2150}
	\maketitle
	
	
	\section*{Scopo dell'~esperienza}
	Misurare le caratteristiche di circuiti lineari realizzati con un op-amp TL081 alimentati tra +15 V e -15 V.
	
	\section{Amplificatore invertente}
	Si vuole realizzare un amplificatore invertente con un'~impedenza di ingresso superiore a 1 
	k$\Omega$ e con un amplificazione a centro banda di 10.
	
	\subsection{Scelta dei componenti}
	
	Si monta il circuito secondo lo schema mostrato in figura \ref{fig:ampinv}, utilizzando la barra di 
	distribuzione verde per la tensione negativa, quella rosso per la tensione positiva, e quella nera per 
	la massa.
	
	\rem{Indicare i criteri di scelta delle resistenze ed i valori desiderati}
	%
	\begin{figure}[h]
		\begin{center}
			%\includegraphics[width=0.4\linewidth]{schema-inv.png}
			\framebox(200,200){Inserire schema circuitale}
			\caption{\small Schema di un amplificatore invertente}
			\label{fig:ampinv}
		\end{center}
	\end{figure}
	%
	
	Le resistenze selezionate hanno i seguenti valori, misurati con il multimetro digitale, con il corrispondente valore atteso 
	del guadagno in tensione dell'~amplificatore.
	\[
	R_1 = ( 1.466 \pm 0.012) \,\mathrm{k}\Omega, \quad 
	R_2 = (15.24  \pm 0.12) \,\mathrm{k}\Omega, \quad 
	A_{exp} = ( -10.39 \pm 0.11)
	\]
	
	\subsection{Montaggio circuito}
	
	%%%%%%%%%%%%%%%%%%%%%%%%%%%%%%%%%%%%%%%%%%%%%%%%%%%%%%
	\subsection{Linearit\`a e misura del guadagno}
	Si fissa la frequenza del segnale ad $f_{in} = (2.597 \pm 0.011)$ kHz e si invia all'~ingresso dell'~amplificatore.	L'uscita dell'~amplificatore \`e mostrata qualitativativamente in Fig. \ref{fig:oscinv} per due 
	differenti ampiezze di $V_{in}$ (circa $xxx$~Vpp e $xxx$~Vpp). 
	Nel primo caso l'~OpAmp si comporta in modo lineare mentre nel secondo caso si osserva clipping. Il datasheet riporta uno Slew rate di $13 V/\mu s$ che è quindi trascurabile a questa frequenza fino ad un'ampiezza di circa 103 V.
	%
	\begin{figure}[h]
		\begin{center}
			\framebox(200,200){Screenshot oscillografo con $V_{out}$ lineare}
			\framebox(200,200){Screenshot oscillografo con clipping di $V_{out}$}
			%\includegraphics[0.45\textwidth]{}
			%\includegraphics[0.45\textwidth]{}
		\end{center}
		\caption{\small Ingresso (in alto) ed uscita (in basso) di un amplificatore invertente con OpAmp, in 
			zona lineare (a sinistra) e non (a destra)}
		\label{fig:oscinv}
	\end{figure}
	%
	
	Variando l'~ampiezza di $V_{in}$ si misura $V_{out}$ ed il relativo guadagno $A_V=V_{out}/V_{in}$ riportando i dati ottenuti in tabella~\ref{tab:guadagno} 
	e mostrandone un grafico in Fig. \ref{fig:lin}. 
	
	\begin{table}[h]
		\caption{$V_{out}$ in funzione di $V_{in}$ e relativo rapporto.}
		\label{tab:guadagno}
		\begin{center}
			\begin{tabular}{|c|c|c|}
				\hline
				$V_{in}$ (V) & $V_{out}$ (V)  & $A_V$ \\
				\hline
				\hline
				$\exn \pm \exn $ & $\exn \pm \exn $ & $\exn \pm \exn$ \\
				\hline
				$\exn \pm \exn $ & $\exn \pm \exn $ & $\exn \pm \exn $ \\
				\hline
				$\exn \pm \exn $ & $\exn \pm \exn $ & $\exn \pm \exn $ \\
				\hline
				$\exn \pm \exn $ & $\exn \pm \exn $ & $\exn \pm \exn $ \\
				\hline
				$\exn \pm \exn $ & $\exn \pm \exn $ & $\exn \pm \exn $ \\
				\hline
			\end{tabular}
		\end{center}
	\end{table}
	
	\rem{Indicare in che modo si fa il fit, se sulla retta $V_{out}$ vs. $V_{in}$ oppure sui valori di $A_V$   }
	
	Si determina il guadagno mediante fit dei dati ottenuti:
	\[
	A_{best} = \exn \pm \exn \quad  \chi^2 = \exn
	\]
	\begin{figure}[t]
		\begin{center}
			\framebox(200,200){Inserire grafico con di $V_{out}$ e $V_{in}$}
			%\includegraphics[0.8\textwidth]{}
		\end{center}
		\caption{\small Linearit\`a dell'~amplificatore invertente}
		\label{fig:lin}
	\end{figure}
	%
	
	\rem{Fino a quale tensione il circuito si comporta linearmente? Provare (facoltativamente) a ridurre la 
		tensione di alimentazione dell'~integrato ed a verificarne la correlazione con la tensione di 
		\emph{clipping} dell'~uscita. Commentare quanto osservato }
	
	%%%%%%%%%%%%%%%%%%
	%
	\section{Risposta in frequenza e \emph{slew rate}}
	\subsection{Risposta in frequenza del circuito}
	Si misura la risposta in frequenza del circuito, riportando i dati  in Tab. \ref{tab:bodeinv} e
	in un grafico di Bode in Fig. \ref{fig:bodeinv}, stimando la frequenza di taglio inferiore e 
	superiore \rem{indicare in che modo}.
	\[
	V_{in} = (\exn \pm \exn )\,\mathrm{V}
	\]
	\[
	f_L = (\exn \pm \exn )\,\mathrm{Hz}\;\;\;\;\;f_H = (\exn \pm \exn \;)\,\mathrm{kHz}
	\]
	\begin{table}[h]
		\caption{\small Guadagno dell'~amplificatore invertente in funzione della frequenza.}
		\label{tab:bodeinv}
		\begin{center}
			\begin{tabular}{|c|c|c|}
				\hline
				$f_{in}$ (kHz) & $V_{out}$ (V) & $A$ (dB) \\
				\hline
				$\exn \pm \exn $ & $\exn \pm \exn $ & $\exn \pm \exn $\\
				\hline
				$\exn \pm \exn $ & $\exn \pm \exn $ & $\exn \pm \exn $\\
				\hline
				$\exn \pm \exn $ & $\exn \pm \exn $ & $\exn \pm \exn $\\
				\hline
				$\exn \pm \exn $ & $\exn \pm \exn $ & $\exn \pm \exn $\\
				\hline
				$\exn \pm \exn $ & $\exn \pm \exn $ & $\exn \pm \exn $\\
				\hline
				$\exn \pm \exn $ & $\exn \pm \exn $ & $\exn \pm \exn $\\
				\hline
			\end{tabular}
		\end{center}
	\end{table} 
	
	
	
	\begin{figure}[h]
		\begin{center}
			\framebox(200,200){Inserire plot di Bode dell'~invertente.}
			%\includegraphics[width=0.7\textwidth]{}
			\caption{\small Plot di Bode in ampiezza per l'~amplificatore invertente.}
			\label{fig:bodeinv}
		\end{center}
	\end{figure}
	%
	\subsection{Misura dello \emph{slew-rate}}
	Si misura direttamente lo \emph{slew-rate} dell'op-amp inviando in ingresso un'~onda quadra 
	di frequenza di $\sim xxx$~kHz e di ampiezza $\sim xxx$~V. Si ottiene:
	\[
	SR_\mathrm{misurato} = (\exn \pm \exn )\,\mathrm{V/\mu s} \quad \mathrm{valore \; tipico}\, (\exn )\,\mathrm{V/\mu s}\
	\]
	
	\rem{Commentare accordo o disaccordo. Eventualmente inserire screenshot dell'oscilloscopio}
	%
	\section{Circuito integratore}
	Si monta il circuito integratore con i seguenti valori  dei componenti indicati: 
	\[
	R_1 = (\exn \pm \exn \;) \,\mathrm{k}\Omega, \:\:\;\:\exn 
	R_2 = (\exn \pm \exn \;) \,\mathrm{k}\Omega, \:\:\;\:\exn 
	C = (\;\exn \pm \exn \;\;)\,\mathrm{nF}
	\]
	
	\subsection{Risposta in frequenza}
	
	Si invia un'~onda sinusoidale e si misura la risposta in frequenza dell'~amplificazione e della fase riportandoli 
	nella tabella \ref{tab:bodeinte} e in un diagramma di Bode in Fig. \ref{fig:bodeinte}. 
	\[
	V_{in} = (\exn \pm \exn )\,\mathrm{V}
	\]
	\rem{La fase pu\'o essere indicata in gradi, radianti, oppure come frazione $\phi/2\pi$}
	%
	\begin{table}[h]
		\caption{Guadagno e fase dell'~integratore invertente in funzione della frequenza.}
		\label{tab:bodeinte}
		\begin{center}
			\begin{tabular}{|c|c|c|c|c|}
				\hline
				$f_{in}$ (kHz) & $V_{out}$ (V) & $A$ (dB) & $\Delta t (\mu s)$ & $\phi$ \\
				\hline
				$\exn \pm \exn $ & $\exn \pm \exn $ & $\exn \pm \exn $ & $\exn \pm \exn $ & $\exn \pm \exn $ \\
				\hline
				$\exn \pm \exn $ & $\exn \pm \exn $ & $\exn \pm \exn $ & $\exn \pm \exn $ & $\exn \pm \exn $ \\
				\hline
				$\exn \pm \exn $ & $\exn \pm \exn $ & $\exn \pm \exn $ & $\exn \pm \exn $ & $\exn \pm \exn $ \\
				\hline
				$\exn \pm \exn $ & $\exn \pm \exn $ & $\exn \pm \exn $ & $\exn \pm \exn $ & $\exn \pm \exn $ \\
				\hline
				$\exn \pm \exn $ & $\exn \pm \exn $ & $\exn \pm \exn $ & $\exn \pm \exn $ & $\exn \pm \exn $ \\
				\hline
				$\exn \pm \exn $ & $\exn \pm \exn $ & $\exn \pm \exn $ & $\exn \pm \exn $ & $\exn \pm \exn $ \\
				\hline
				$\exn \pm \exn $ & $\exn \pm \exn $ & $\exn \pm \exn $ & $\exn \pm \exn $ & $\exn \pm \exn $ \\
				\hline
			\end{tabular}
		\end{center}
	\end{table} 
	%
	\begin{figure}[htb]
		\begin{center}
			\framebox(200,200){Inserire plot di Bode in ampiezza}
			\framebox(200,200){Analogo in fase}
			%\includegraphics[0.45\textwidth]{}
			%\includegraphics[0.45\textwidth]{}
		\end{center}
		\caption{\small Plot di Bode in ampiezza (a sinistra) e fase (a destra) per il circuito integratore.}
		\label{fig:bodeinte}
	\end{figure}
	%
	
	Si ricava una stima delle caratteristiche principali dell'andamento (guadagno a bassa frequenza, frequenza di taglio, e pendenza ad alta frequenza)
	e si confrontano con quanto atteso. Non si effettua la stima degli errori, trattandosi di misure qualitative.
	
	\rem{Indicare brevemente come sono stati ottenuti i valori attesi}
	
	\begin{align*}
	A_M &= (\exn )\,\mathrm{dB} & \mathrm{atteso} &:\,(\exn  )\, \mathrm{dB}  \\
	f_H &= (\exn )\,\mathrm{Hz} & \mathrm{atteso} &:\,(\exn  )\, \mathrm{Hz} \\
	{\mathrm{d}A_V}/{\mathrm{d}f} &= (\exn )\,\mathrm{dB/decade} & \mathrm{atteso} &:\,(\exn  )\, \mathrm{dB/decade}  \\
	\end{align*}
	
	
	%
	\subsection*{Risposta ad un'~onda quadra}
	Si invia all'~ingresso un'~onda quadra di frequenza $\sim xxx\,kHz$ e ampiezza $\sim xxx\,V$.
	Si riporta in Fig. \ref{fig:oscinte} le forme d'~onda acquisite all'~oscillografo per l'~ingresso
	e l'~uscita. 
	
	\rem{Commentare se che il circuito si comporta come un integratore.}
	%
	\begin{figure}[htb]
		\begin{center}
			\framebox(200,200){Inserire screenshot oscillografo per integratore}
			%\includegraphics[0.45\textwidth]{}
		\end{center}
		\caption{\small Ingresso (in alto) ed uscita (in basso) del circuito integratore per un'~onda quadra.}
		\label{fig:oscinte}
	\end{figure}
	%
	
	Si misura l'~ampiezza dell'~onda  in uscita e si confronta il valore atteso.
	
	\rem{Indicare brevemente come sono stati ottenuti i valori attesi}
	\begin{align*}
	V_{out} &= (\exn )\,\mathrm{V} & \mathrm{atteso} &:\,(\exn  )\, \mathrm{V}  \\
	\end{align*}
	
	\rem{Inserire commento sulla dipendenza dell'~uscita dalla frequenza.}
	%
	
	\subsection{Discussione}
	
	\rem{Inserire commenti su quanto osservato ed eventuali deviazioni. 
		In particolare: attenuazione ad alte frequenze, dipendenza della fase dalla frequenza, funzione di $R_2$. }
	
	%%%%%%%%%%%%%%%%%%%%%%%%
	
\end{document}          
