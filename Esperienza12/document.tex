\documentclass[10pt,a4paper]{article}
\usepackage[utf8]{inputenc}
\usepackage[italian]{babel}
\usepackage{amsmath}
\usepackage{amsfonts}
\usepackage{amssymb}
\usepackage{graphicx}
\usepackage{siunitx}
\usepackage[left=2cm,right=2cm,top=2cm,bottom=2cm]{geometry}
\newcommand{\rem}[1]{[\emph{#1}]}
\newcommand{\exn}{\phantom{xxx}}
\usepackage[italian]{babel}
\usepackage[utf8]{inputenc}
\usepackage{siunitx}
\usepackage{graphicx}
\usepackage{xcolor}
\usepackage{amsfonts}
\usepackage{amsmath}
\usepackage{amsthm}
\usepackage{tikz}
\usepackage{pgfplots}
\usepackage{enumitem}
\usepackage{subcaption}
\usepackage{siunitx}
\date{\today}
\usetikzlibrary{shapes.geometric,calc,matrix,arrows,snakes,shapes,patterns}
\title{Esercitazione 12B: Macchine e stati finiti:semaforo}
\author{Massimo Bilancioni, Alessandro Foligno, Giuseppe Zanichelli}
\begin{document}	
\maketitle
Si decide arbitrariamente di assegnare vari stati di $Q_0$ e $Q_1$ ad altrettante posizioni del semaforo. In particolare, la corrispondenza scelta è:\\
\begin{enumerate}
	\item Lo stato (0,0) corrisponde al Verde
	\item Lo stato (0,1) corrisponde al Giallo+Verde
	\item Lo stato (1,0) corrisponde al Rosso
	\item Lo stato (1,1) corrisponde ad uno stato a caso (non fa parte del ciclo)
\end{enumerate}
Utilizzando questa corrispondenza, lo schema della transizione è riportato nella seguente tabella (insieme con i valori corrispondenti dei led)\
\begin{table}[h]\centering
\begin{tabular}{|c|c|c|c|c|c|c|}
	\hline 
	$Q_0^n$ & $Q_1^n$ & $Q_0^{n+1}$ & $Q_1^{n+1}$ & Verde & Giallo & Rosso \\ 
	\hline 
	0 & 0 & 0 & 1 & 1 & 0 & 0 \\ 
	\hline 
	0 & 1 & 1 & 0 & 1 & 1 & 0 \\ 
	\hline 
	1 & 0 & 0 & 0 & 0 & 0 & 1 \\ 
	\hline 
	1 & 1 & x & x & x & x & x \\ 
	\hline 
\end{tabular} 	
\caption{Tabella di transizione per $Q_0$ e $Q_1$}
\end{table}
\\
osservando la tabella si vede che il segnale su Verde è proprio $\bar{Q_0}$, quello su Giallo è $Q_1$, quello su Rosso è $Q_0$.
Si riporta lo schema del circuito realizzato.








\end{document}