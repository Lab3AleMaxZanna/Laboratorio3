\documentclass[10pt,a4paper]{article}
\usepackage[utf8]{inputenc}
\usepackage[italian]{babel}
\usepackage{amsmath}
\usepackage{amsfonts}
\usepackage{amssymb}
\usepackage{graphicx}
\usepackage{siunitx}
\usepackage[left=2cm,right=2cm,top=2cm,bottom=2cm]{geometry}
\newcommand{\rem}[1]{[\emph{#1}]}
\newcommand{\exn}{\phantom{xxx}}
\usepackage[italian]{babel}
\usepackage[utf8]{inputenc}
\usepackage{siunitx}
\usepackage{graphicx}
\usepackage{xcolor}
\usepackage{amsfonts}
\usepackage{amsmath}
\usepackage{amsthm}
\usepackage{tikz}
\usepackage{pgfplots}
\usepackage{enumitem}
\usepackage{subcaption}
\usepackage{siunitx}
\date{\today}
\usetikzlibrary{shapes.geometric,calc,matrix,arrows,snakes,shapes,patterns}
\title{Esercitazione 11B: Flip-Flop e contatori}
\author{Massimo Bilancioni, Alessandro Foligno, Giuseppe Zanichelli}
\begin{document}	
\maketitle
	
\section{ Flip-Flop D-Latch}
\subsection{Montaggio}
 Si monta il circuito come in Figura \ref{circ}; per attivare a piacimento l'Enable lo si collega con un interruttore a terra in modo tale che quando l'interruttore è aperto, Enable è in alto, quando è chiuso Enable è a terra.
 \begin{figure}[h]
 	\label{circ}
 	\includegraphics[scale=1]{circuito.png}

 	\caption{schema del circuito realizzato}
 \end{figure}
\subsection{Tempi di ritardo}
Si verifica che quando l'interruttore è aperto ($En=1$), il circuito copia il valore di $D$ su $Q$, nel caso opposto, il valore di $Q$ resta costante. Tenendo il valore di $En$ alto, si manda in ingresso su $D$ un'onda quadra, per misurare il tempo di ritardo. \\Le misure ricavate sono(i segnali osservati sono riportati in Figura \ref{dlatch}):\\ $\Delta t_{salita}\approx40 ns$
\\$\Delta t_{discesa}\approx60 ns$\\
\begin{figure}[h]

	\includegraphics[scale=0.7]{ritardo2.png}
	\includegraphics[scale=0.7]{ritardodiscesa2.png}
	\caption{D-Latch con Enable on e in ingresso un'onda quadra: tempo di ritardo in salita e discesa}
		\label{dlatch}
\end{figure} 
La differenza di tempi fra salita e discesa può essere spiegata dando un'occhiata al circuito. Quando $D$ va in alto, il nuovo segnale deve passare attraverso il primo NAND e dare FALSE, dopodichè il NAND successivo restituirà immediatamente TRUE, indipendentemente dal segnale su $\bar{Q}$. Invece, quando $D$ diventa FALSE,il NAND che restituisce $Q$ avrà ad un ingresso TRUE e all'altro $\bar{Q}$, quindi affinchè il segnale su $Q$ sia quello giusto, deve prima arrivare il segnale corretto su $\bar{Q}$ e quindi bisogna passare attraverso altri due NAND, con un ritardo di circa $10 ns$ ciascuno. Questo spiega anche perchè il segnale sull'oscilloscopio(arancione) è così diverso da quello in ingresso (quello blu).

\textcolor{red}{NON SO SE DEVI ANCORA FINIRE IN OGNI CASO AGGIUNGI L'ENABLE LA FIGURA}
l'immagine \textcolor{red}{AGGIUNGERE IMMAGINE}.
Si è verificato il corretto funzionamento del contatore con un impulso a bassa frequenza.                                                                                                                                                                                                                                                   

Abbiamo inviato un clock a una frequenza di circa $f\simeq 50 \si{\kilo \hertz}$; nelle immagini che seguono compaiono  rispettivamente $Q0$, $Q1$, $Q2$ e $Q3$ \textcolor{red}{AGGIUNGERE IMMAGINE} (in giallo) sovrapposti al clock (in blu), dalle figure si vede che le frequenze sono effettivamente $1/2$, $1/4$, $1/8$, $1/16$ di $f$.
\begin{figure}[h]
			\centering
			\includegraphics[scale=0.85]{1mezzo}
			\caption{sengale $Q0$ in blu}
			\label{fig:plh}
\end{figure}
\begin{figure}[h]
			\centering
			\includegraphics[scale=0.85]{1quarto}
			\caption{sengale $Q1$ in blu}
			\label{fig:plh}
\end{figure}
\begin{figure}[h]
			\centering
			\includegraphics[scale=0.85]{1ottavo}
			\caption{sengale $Q2$ in blu}
			\label{fig:plh}
\end{figure}
Tramite l'oscilloscopio si è poi misurato il ritardo tra la transizione del clock e quella di $Q0$, $Q1$, $Q2$ e $Q3$; in particolare si è misurato l'intervallo temporale che intercorreva da quando la tensione in ingresso passava per il $50\%$ del valore massimo a quando l'uscita passava per  il $50\%$ del valore massimo. Nelle Figure \ref{fig:t1} e \ref{fig:t2} sono mostrati il segnale di clock (in blu) e $Q0$ (in giallo) rispettivamente quando il segnale  $Q0$ passa da alto a basso e da basso ad alto. I ritardi misurati per questa uscita 
sono nel primo caso $t\ped{discesa} = (150 \pm 5)\si{\nano\second}$ e nel secondo  $t\ped{salita} = (30 \pm 5)\si{\nano\second}$.\textcolor{red}{CONTROLLARE COERENZA ERRORI NEI RITARDI PUNTO 2-3}
Per gli altri tre i ritardi sono gli stessi di $Q0$  nei due casi, infatti le quattro uscite sono quasi perfettamente sincrone tra loro; come esempio  mostriamo  i segnali $Q0$ e $Q1$ nella transizione da alto a basso in Figura \ref{fig:q0q1}.
\textcolor{red}{SINCRONIA TRA TRANSIZIONE MISTA?BOH!}
\textcolor{red}{CONFRONTARE TEMPI CON DATASHIIIIIT NON TORNA NULLA!}
\begin{figure}[h]
			\centering
			\includegraphics[scale=0.85]{tcounter}
			\caption{ritardo di $Q0$ nel passare da alto a basso}
			\label{fig:t1}
\end{figure}
\begin{figure}[h]
			\centering
			\includegraphics[scale=0.85]{tcounter_1}
			\caption{ritardo di $Q0$ nel passare da basso ad alto}
			\label{fig:t2}
\end{figure}

\begin{figure}[h]
			\centering
			\includegraphics[scale=0.85]{Q0Q1sincroni}
			\caption{quasi perfetta sincronia tra i segnali $Q0$ e $Q1$ quando entrambi diventano bassi}
			\label{fig:q0q1}
\end{figure}
\section{Shift register con D-Latch}
\subsection{Montaggio e funzionamento}
Il circuito è stato montato come in figura. I pin CLR sono stati collegati tramite una resistenza di pull-up ai 5V. Il pulsante è stato usato come un deviatore, in modo da riferire anche i preset.
\begin{figure}[h]
	\includegraphics[scale=1]{circuito3.png}
	\caption{Shiftregister}                                
\end{figure}
I led si accendono in sequenza, partendo dal primo e scorrendo fino all'ultimo, l'accensione di uno sfasata di un periodo di clock rispetto all'accensione del successivo. Il primo segue il DIP switch, con un ritardo di al massimo un periodo di clock. Dopo aver premuto il bottone di preset le luci sono tutte accese (il pin è negato)

\subsection{Feedback negato}
Collegando l'uscita negata Q3 all'input del primo FF il circuito inizia a oscillare con una frequenza pari a un quarto di quella di clock. I led si accendono in successione da Q0 a Q3 per poi spegnersi nello stesso ordine

\begin{figure}[h]
	\centering
	\begin{subfigure}[b]{0.4\linewidth}
		\includegraphics[width=\linewidth]{q0q1periodosr.png}
		\caption{Confronto tra Q0 e Q1, visibile più di un periodo}
	\end{subfigure}
	\begin{subfigure}[b]{0.4\linewidth}
		\includegraphics[width=\linewidth]{q0q1sr.png}
		\caption{Confronto tra Q0 e Q1, sfasamento alla salita}
	\end{subfigure}
	\newline
	\begin{subfigure}[b]{0.4\linewidth}
		\includegraphics[width=\linewidth]{q0q2sr.png}
		\caption{Q0 e Q2, sfasamento alla salita}
	\end{subfigure}
	\begin{subfigure}[b]{0.4\linewidth}
		\includegraphics[width=\linewidth]{q0q3sr.png}
		\caption{Q0 e Q3, sfasamento alla salita}
	\end{subfigure}
	\caption{Shift register con feedback negato}
\end{figure}

Nelle figure si può  vedere un confronto tra le onde quadre emesse su ogni piedino e Q0. Si può osservare facilmente lo sfasamento di 1,2,3 tempi di clock.

\end{document}